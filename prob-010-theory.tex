\documentclass{article}
\usepackage[margin=1in]{geometry}
\usepackage{graphicx}
\usepackage[toc,page]{appendix}
\usepackage{hyperref}
\usepackage{amsmath,amsfonts,amssymb,mathrsfs,amsthm}
\usepackage{multirow}
\usepackage{placeins}
\usepackage{subcaption}
%\usepackage{titlesec}

%\setcounter{secnumdepth}{4}

%\titleformat{\paragraph}
%{\normalfont\smallsize\bfseries}{\theparagraph}{1em}{}
%\titlespacing*{\paragraph}
%{0pt}{3.25ex plus 1ex minus .2ex}{1.5ex plus .2ex}

% Default fixed font does not support bold face
\DeclareFixedFont{\ttb}{T1}{txtt}{bx}{n}{12} % for bold
\DeclareFixedFont{\ttm}{T1}{txtt}{m}{n}{12}  % for normal

\DeclareMathOperator*{\argmin}{arg\,min}  

% Custom colors
\usepackage{color}
\definecolor{deepblue}{rgb}{0,0,0.5}
\definecolor{deepred}{rgb}{0.6,0,0}
\definecolor{deepgreen}{rgb}{0,0.5,0}

\usepackage{listings}
\usepackage{courier}
\usepackage[parfill]{parskip}


\newcommand*\conj[1]{\overline{#1}}
\newcommand\floor[1]{\lfloor#1\rfloor}
\newcommand\ceil[1]{\lceil#1\rceil}

\newcommand\TheSolution{
  \mbox{}\par\nobreak
  \noindent
  \textbf{Solution:}\\
}

\newtheorem{theorem}{Theorem}
\newtheorem{case}{Case}
\newtheorem{lemma}{Lemma}

%\theorempostwork{\setcounter{case}{0}}

\makeatletter
\@addtoreset{case}{theorem}
\@addtoreset{case}{lemma}
\makeatother

% (Don't reset lemmas)
%\makeatletter
%\@addtoreset{lemma}{theorem}
%\makeatother

\setcounter{secnumdepth}{0}


\title{Tales of Probability, Part 1}
\author{Kevin Yuh}

\begin{document}
\maketitle


% need a better title for this section
\section{Introduction}

What is probability? It can be hard to put a finger on the whole concept; after all, it is a theory of uncertainty in a world where concrete, certain things tend to happen. Yet, we've all run into scenarios when it comes into play - what is the chance I get a 6 from a die roll? What is the chance of winning the lottery? What is the chance that I live to seventy years old?
% more here

% stuff here

More importantly, how do we describe probability? 

We often intuitively know the chances of events in simple scenarios. If I have a die, with sides labeled 1, 2, 3, 4, 5, and 6, we intuitively know that the probability of rolling a 3 is a 1/6 chance. 


\subsection{Meaning of the generating function here}


\subsection{Further work}





\end{document}
