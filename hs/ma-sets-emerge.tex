%!TEX root = tales-main.tex


% Big theme here: Emergent properties
%	(i.e. what happens when we have this property (here, among members of sets)
%	we can get some interesting consequence (e.g. dimension))

\chapter{(Sets and Emergence)}

% Sth: "This might be the most boring way to start a chapter. In the past, we've talked about counting things -- here, we'll be talking about, quite literally, "things" (or more properly, collections of things)"


\section{(Sets -- collections of things)}

\begin{exercise}
What are the intersection and union of...
\begin{enumerate}[(a)]
\item ...the set $\{1, 2, 3\}$ and the set $\{2, 3, 4\}$?
\item ...the "null set" (the set with nothing in it) and any other set?
\item ...the set of all even integers, and the set of all odd integers?
\end{enumerate}
\end{exercise}

\begin{exercise}
What is the intersection of...
\begin{enumerate}[(a)]
\item ...the set of prime numbers, and the set of even numbers?
\item ...the set of integers divisible by 2, and the set of integers divisible by 3?
\end{enumerate}
\end{exercise}

% TODO: Need to have defined prime numbers at some point

\section{Problems}

\begin{problem}
\label{pr:amoeba-basic}
Suppose we have an amoeba. It has a probability $p_d$ of dying, and a probability $(1 - p_d)$ of dividing into two amoebas (which themselves have the same probabilities of dying and dividing). Given our one starting amoeba, what is the probability that the whole "family tree" dies out?
\end{problem}
% TODO: Insert solution (and revise it for rigor)
% TODO: Include another problem as an extension, where "magical amoeba" can divide into n copies of itself. Use as interesting plotting/limiting behavior exercise (connect to "yeah as n grows, less likely to die")
% TODO: Problem/extension - consider what happens when we have more than one starting amoeba

\newpage
\section{Terms and References}

\section{Hints}

\newpage
\section{Solutions}

\subsection{Problem \ref{pr:amoeba-basic}}

Let $p_G$ be the probability of the generation dying out, starting from one amoeba. There are two possibilities for that amoeba's family tree to die off: 

\begin{enumerate}
\item The amoeba itself dies. (This occurs with probability $p_d$.)
\item The amoeba survives, but dies off in future generations. The probability of the amoeba surviving (denote $P(parent survival)$) is $(1 - p_d)$. The probability of the generation dying off, given survival of the parent amoeba, is $P(gen survival | parent survival) = p_G^2$; this is because the probability of a generation dying, starting from a given amoeba, is $p_G$, and we have to have this occur independently on each child of the original parent. 

Hence, the probability of this whole chain of events (parent survival and gen survival) is the product of the two quantities above (since $P(A \cap B) = P(B | A)P(A)$), and we get $(1-p_d)p_G^2$. 
\end{enumerate}

Hence, we have the equation

\begin{align*}
p_G = p_d + (1-p_d)p_G^2
\end{align*}

Solving this gives $p_G = p_d / (1 - p_d)$ and $p_G = 1$. Because of (FILL IN HERE), the latter is of course invalid. 

For example, if $p_d = 1/4$, then $p_G = 1/3$. % TODO: In other words...

$\square$

% TODO: Be more rigorous on why $p_G = 1$ is not a valid solution?
% TODO: Make this more rigorous with probability spaces, etc.





