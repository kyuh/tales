%!TEX root = tales-main.tex

\chapter{(Calc - Differentiation)}


% Central question: What is the slope of this line?
% (Insert figure of tangent at x^2 for some positive val)
%	(Include the graph lines in this one?, but point to/highlight the tangent somehow)
% ACTUALLY, maybe we shouldn't have this figure at the start - see approach below.






% TODO: Should this chapter have a broader intro?


\section{A Line's Rise and Fall}

% Lines have slope
%	- We often find it by saying - if there are two points on the line, then we can see how much 
% Next section: How can we find the slope of *this* (tangent) line?

Consider a line on a graph:

% TODO: Insert figure of single line on graph

One of its major features is its \textbf{slope} -- how fast does it rise or fall? Some lines, like the dark lines below, are nearly flat -- but some lines are very "steep". In the figure below, the red lines "rise" very quickly, while the blue lines rapidly "fall".

% TODO: Insert a figure with many generated lines (not necessarily all with same y-intercept). Establish a gradient red -> black -> blue by slope, as described above.

How we can state this more precisely? A line is a function -- let's call it $l$, so $l(x)$ is the value of the function at some $x$.

If a line is "rising" rapidly, then $l(x)$ increases a lot when we increase its $x$-value. If it's falling rapidly, then $l(x)$ \textit{decreases} quite a bit when we increase $x$. And for a line that's relatively flat, $l(x)$ will change slowly, even as we change the $x$-value quite a bit.

% TODO: Illustrate rising line with a big jump in y relative to a jump in x


What's a more accurate way of defining slope? 

If we know two points on the line, then we can 



\newpage
\section{Terms and References}

\section{Solutions}

\subsection{Potential exercise (derivative of $x^n$ for integer $n$)}

\begin{align*}
\frac{d}{dx}x^n &= \lim_{h \rightarrow 0} \frac{(x+h)^n - x^n}{h} \\
&= \lim_{h \rightarrow 0} \frac{\sum_{k=0}^n \binom{n}{k}h^k x^{n-k} - x^n}{h} && \text{(Binomial expansion -- see chapter)} \\
&= \lim_{h \rightarrow 0} \frac{\binom{n}{0}x^n + \binom{n}{1}hx^{n-1} + \sum_{k=2}^n \binom{n}{k}h^k x^{n-k} - x^n}{h} && \text{(Separate $k=0$ and $k=1$ terms)} \\
&= \lim_{h \rightarrow 0} \frac{x^n + nhx^{n-1} + \sum_{k=2}^n \binom{n}{k}h^k x^{n-k} - x^n}{h} \\
&= \lim_{h \rightarrow 0} \frac{nhx^{n-1} + \sum_{k=2}^n \binom{n}{k}h^k x^{n-k}}{h} \\
&= \lim_{h \rightarrow 0} \left[nx^{n-1} + \sum_{k=2}^n \binom{n}{k}h^{k-1} x^{n-k}\right] \\
&= nx^{n-1} + 0 && \text{($k-1$ is positive at all terms, so all $h^{k-1}$ terms approach 0)}\\
\end{align*}
