% Notes to authors:
%
% - Don't be afraid to just start writing, you'll probably
% 	have to rewrite most of it anyway.
% 
% - We can't introduce all of science in this book, and it's
% 	meant to give a taste of thinking like a scientist.
% 	Know what to leave out. 


% Thoughts on style:
% - Use shorter paragraphs, when possible. They're easier to read.

\documentclass{article}
\usepackage[margin=2in]{geometry}
\usepackage{graphicx}
\usepackage[toc,page]{appendix}
\usepackage{hyperref}
\usepackage{amsmath,amsfonts,amssymb,mathrsfs,amsthm}
\usepackage{multirow}
\usepackage{placeins}
\usepackage{subcaption}
%\usepackage{titlesec}

%\setcounter{secnumdepth}{4}

%\titleformat{\paragraph}
%{\normalfont\smallsize\bfseries}{\theparagraph}{1em}{}
%\titlespacing*{\paragraph}
%{0pt}{3.25ex plus 1ex minus .2ex}{1.5ex plus .2ex}

% Default fixed font does not support bold face
\DeclareFixedFont{\ttb}{T1}{txtt}{bx}{n}{12} % for bold
\DeclareFixedFont{\ttm}{T1}{txtt}{m}{n}{12}  % for normal

\DeclareMathOperator*{\argmin}{arg\,min}  

% Custom colors
\usepackage{color}
\definecolor{deepblue}{rgb}{0,0,0.5}
\definecolor{deepred}{rgb}{0.6,0,0}
\definecolor{deepgreen}{rgb}{0,0.5,0}

\usepackage{listings}
\usepackage{courier}
\usepackage[parfill]{parskip}


\newcommand*\conj[1]{\overline{#1}}
\newcommand\floor[1]{\lfloor#1\rfloor}
\newcommand\ceil[1]{\lceil#1\rceil}

\newcommand\TheSolution{
  \mbox{}\par\nobreak
  \noindent
  \textbf{Solution:}\\
}

\newtheorem{theorem}{Theorem}
\newtheorem{case}{Case}
\newtheorem{lemma}{Lemma}
\newtheorem{exercise}{\\ \bf Exercise}

%\theorempostwork{\setcounter{case}{0}}

\makeatletter
\@addtoreset{case}{theorem}
\@addtoreset{case}{lemma}
\makeatother

% (Don't reset lemmas)
%\makeatletter
%\@addtoreset{lemma}{theorem}
%\makeatother

%\setcounter{secnumdepth}{0}


\title{Tales of Sums (Art of Science)}
\date{\vspace{-5ex}}

\begin{document}
\maketitle


% Central question: (Have one)

\section{Introduction}


\section{Terms and References}

\section{Problems}
% This section is kind of just an agglomeration of them for now

\begin{exercise}
\label{power-2-sum}
% TODO: Give lead-up

Consider the following pattern:

\begin{align*}
1 + 1 &= 2 \\
1 + 1 + 2 &= 4 \\
1 + 1 + 2 + 4 &= 8 \\
1 + 1 + 2 + 4 + 8 &= 16 \\
...
\end{align*}


Another way of looking at it:

\begin{align*}
1 + 2^0 &= 2^1 \\
1 + 2^0 + 2^1 &= 2^2 \\
1 + 2^0 + 2^1 + 2^2 &= 2^3 \\
1 + 2^0 + 2^1 + 2^2 + 2^3 &= 2^4 \\
...
\end{align*}


Essentially, by summing up a bunch of powers of 2 in sequence (starting from 1), one just *almost* reaches the next power -- all we need is to add an extra 1!

% SCRAP: Essentially, by summing up a bunch of powers of 2 in sequence (starting from 1), one just *almost* reaches the next one!

Let's prove it. Prove that this pattern does exist for all integer powers of 2 -- in other words, prove that $1 + \sum_{i=0}^{n-1} 2^i = 2^n$, for all integers $n \geq 1$.

\end{exercise}


\section{Solutions}



\end{document}
