%!TEX root = tales-main.tex

\chapter{(Calc - Differentiation)}


% Central question: What is the slope of this line?
% (Insert figure of tangent at x^2 for some positive val)
%	(Include the graph lines in this one?, but point to/highlight the tangent somehow)
% ACTUALLY, maybe we shouldn't have this figure at the start - see approach below.






% TODO: Should this chapter have a broader intro?


\section{A Line's Rise and Fall}

% Lines have slope
%	- We often find it by saying - if there are two points on the line, then we can see how much 
% Next section: How can we find the slope of *this* (tangent) line?

Consider a line on a graph:

% TODO: Insert figure of single line on graph
\vspace{2in}

One of its major features is its \textbf{slope} -- how fast does it rise or fall? Some lines, like the dark lines below, are nearly flat -- but some lines are very "steep". In the figure below, the red lines "rise" very quickly, while the blue lines rapidly "fall".

% TODO: Insert a figure with many generated lines (not necessarily all with same y-intercept). Establish a gradient red -> black -> blue by slope, as described above.
\vspace{2in}

How we can state this more precisely? A line is a function -- let's call it $l$, so $l(x)$ is the value of the function at some $x$.

If a line is "rising" rapidly, then $l(x)$ increases a lot when we increase its $x$-value. If it's falling rapidly, then $l(x)$ \textit{decreases} quite a bit when we increase $x$. And for a line that's relatively flat, $l(x)$ will change slowly, even as we change the $x$-value quite a bit.

% TODO: Illustrate rising line with a big jump in y relative to a jump in x
\vspace{2in}

Let's say we know of any two points on the line, $(x_1, y_1)$ and $(x_2, y_2)$, where $y_1 = l(x_1)$ and $y_2 = l(x_2)$. Then the quantity:

\begin{equation}
\label{def-slope}
m = \frac{l(x_2) - l(x_1)}{x_2 - x_1}
\end{equation}

basically says "how rapidly $l(x)$ changes relative to $x$". 

We'll define $m$ in Equation \ref{def-slope} as the \textbf{slope}. If $m$ is large and positive, then $l$ is rapidly "rising" -- if $m$ is large and negative, it's rapidly "falling", and a small $m$ (close to 0) would mean an almost flat line.

\begin{exercise}
\label{slope-qns}
Having a consistent slope for a line $l$ only works if $l$ is actually a line. If we tried to use Equation \ref{def-slope} on an arbitrary function $f$, we'd get all sorts of different slopes, depending on what we chose for $x_1$ and $x_2$!

%TODO: Figure with multiple secant lines for f(x) = x^2, let's say going through 1
\vspace{2in}

\begin{enumerate}[(a)]
\item Let's demonstrate this point. Let's take $f(x) = x^2$. Try using $x_1 = 0$ but switching between $x_2 = 1, 2, 3$. What "slopes" do you get?

\item Let's give a brief "proposal" for how a straight line should be defined: We'll say that for \textit{any} two values $x_1$ and $x_2$, we should always get the same slope using Equation \ref{def-slope}. 

Given this proposition, prove that we can always write the function $l$ in the form $l(x) = mx + b$. (Also, can we write $b$ in terms of $l$? Any significance of this value in particular?)
\end{enumerate}
\end{exercise}



\section{The Tangent Line}

% Part (a) of our previous exercise isn't as silly as it seems.

Now consider a slightly different problem. Suppose we have a function that might \textit{not} be a line -- but let's say we want to find its "slope" anyway.

% TODO: Insert simple graph of f(x) = x^2, no tangent line
\vspace{2in}

What does that even mean? 

Let's look at the function $f(x) = x^2$ above. The function seems to be falling and rising, but it looks like we can say, at any particular point on the graph, how fast it's rising or falling. For example, it seems to be falling very rapidly when $x$ is negative, but gradually less rapidly. At $x = 0$, the graph seems to be "flat", and then for positive $x$, the graph seems to be rising -- and then seems to be rising more steeply as $x$ gets larger.

How can we measure that?

% TODO: Insert simple graph of f(x) = x^2, no tangent line
\vspace{2in}

Perhaps the figure above can hint at something. Let's say we can draw a line that seems to "lie" on a point with the same "slope" as the function itself seems to have there. (We'll call this a \textbf{tangent line}.) The function doesn't seem to have a broadly defined slope -- but the line does!

So what is the slope of this line?

There's a slight problem -- in the past, we used two points on a line to calculate the slope. However, we now have only one point!

% TODO: Insert simple graph of f(x) = x^2, with secant lines similar to exercise above, part (a)
\vspace{2in}

But let's think about our "silly"-looking exercise from before. It wasn't sensical to define a slope for a function that wasn't a line, but now we have a curious observation. As we moved $x_2$ closer to $x_1$ (which was 1), our line started to get closer in slope to being a tangent line at $x_1 = 1$!

\subsection{(Limit)}

Let's find out what this slope really is.

For any value $x_2$, we have an "almost-tangent" line defined by the points $(x_1, f(x_2))$ and $(x_2, f(x_2))$. Then its slope is given by

\begin{align*}
m &= \frac{f(x_2) - f(x_1)}{x_2 - x_1} \\
&= \frac{x_2^2 - x_1^2}{x_2 - x_1}
\end{align*}

Here we have $x_1 = 1$, so:

\begin{align*}
m &= \frac{x_2^2 - 1^2}{x_2 - 1} \\
&= \frac{x_2^2 - 1}{x_2 - 1} \\
&= \frac{(x_2 + 1)(x_2 - 1)}{x_2 - 1} \\
\end{align*}


\newpage
\section{Terms and References}

\section{Solutions}

\subsection{Exercise \ref{slope-qns}}

\subsubsection{Part b}

We know that the "slope" is the same for any points $x_1$ and $x_2$. 

Let's pick a particular value for $x_1$ --- say, $x_1 = 0$. Then, no matter what $x_2$ is, we know that:

\begin{equation*}
m = \frac{l(x_2) - l(0)}{x_2 - 0} = \frac{l(x_2) - l(0)}{x_2}
\end{equation*}

and therefore, 

\begin{equation*}
m x_2 = l(x_2) - l(0)
\end{equation*}

and therefore, 

\begin{equation*}
l(x_2) = m x_2 + l(0)
\end{equation*}

We now have the functional form we were looking for, where $b = l(0)$. In other words, $b$ is the value of our function at 0. (This value is often called the "$y$-intercept", because it's the value of the function when it crosses the "$y$-axis" of a graph, i.e. when the function's input is 0.)


\subsection{Potential exercise (derivative of $x^n$ for integer $n$)}

\begin{align*}
\frac{d}{dx}x^n &= \lim_{h \rightarrow 0} \frac{(x+h)^n - x^n}{h} \\
&= \lim_{h \rightarrow 0} \frac{\sum_{k=0}^n \binom{n}{k}h^k x^{n-k} - x^n}{h} && \text{(Binomial expansion -- see chapter)} \\
&= \lim_{h \rightarrow 0} \frac{\binom{n}{0}x^n + \binom{n}{1}hx^{n-1} + \sum_{k=2}^n \binom{n}{k}h^k x^{n-k} - x^n}{h} && \text{(Separate $k=0$ and $k=1$ terms)} \\
&= \lim_{h \rightarrow 0} \frac{x^n + nhx^{n-1} + \sum_{k=2}^n \binom{n}{k}h^k x^{n-k} - x^n}{h} \\
&= \lim_{h \rightarrow 0} \frac{nhx^{n-1} + \sum_{k=2}^n \binom{n}{k}h^k x^{n-k}}{h} \\
&= \lim_{h \rightarrow 0} \left[nx^{n-1} + \sum_{k=2}^n \binom{n}{k}h^{k-1} x^{n-k}\right] \\
&= nx^{n-1} + 0 && \text{($k-1$ is positive at all terms, so all $h^{k-1}$ terms approach 0)}\\
\end{align*}
