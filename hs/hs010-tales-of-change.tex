% Notes to authors:
%
% - Don't be afraid to just start writing, you'll probably
% 	have to rewrite most of it anyway.
% 
% - We can't introduce all of science in this book, and it's
% 	meant to give a taste of thinking like a scientist.
% 	Know what to leave out. 


% Thoughts on style:
% - Use shorter paragraphs, when possible. They're easier to read.

\documentclass{article}
\usepackage[margin=2in]{geometry}
\usepackage{graphicx}
\usepackage[toc,page]{appendix}
\usepackage{hyperref}
\usepackage{amsmath,amsfonts,amssymb,mathrsfs,amsthm}
\usepackage{multirow}
\usepackage{placeins}
\usepackage{subcaption}
%\usepackage{titlesec}

%\setcounter{secnumdepth}{4}

%\titleformat{\paragraph}
%{\normalfont\smallsize\bfseries}{\theparagraph}{1em}{}
%\titlespacing*{\paragraph}
%{0pt}{3.25ex plus 1ex minus .2ex}{1.5ex plus .2ex}

% Default fixed font does not support bold face
\DeclareFixedFont{\ttb}{T1}{txtt}{bx}{n}{12} % for bold
\DeclareFixedFont{\ttm}{T1}{txtt}{m}{n}{12}  % for normal

\DeclareMathOperator*{\argmin}{arg\,min}  

% Custom colors
\usepackage{color}
\definecolor{deepblue}{rgb}{0,0,0.5}
\definecolor{deepred}{rgb}{0.6,0,0}
\definecolor{deepgreen}{rgb}{0,0.5,0}

\usepackage{listings}
\usepackage{courier}
\usepackage[parfill]{parskip}


\newcommand*\conj[1]{\overline{#1}}
\newcommand\floor[1]{\lfloor#1\rfloor}
\newcommand\ceil[1]{\lceil#1\rceil}

\newcommand\TheSolution{
  \mbox{}\par\nobreak
  \noindent
  \textbf{Solution:}\\
}

\newtheorem{theorem}{Theorem}
\newtheorem{case}{Case}
\newtheorem{lemma}{Lemma}

%\theorempostwork{\setcounter{case}{0}}

\makeatletter
\@addtoreset{case}{theorem}
\@addtoreset{case}{lemma}
\makeatother

% (Don't reset lemmas)
%\makeatletter
%\@addtoreset{lemma}{theorem}
%\makeatother

%\setcounter{secnumdepth}{0}


\title{Tales of Change (Art of Science)}
\date{\vspace{-5ex}}

\begin{document}
\maketitle


% Central question: What is the area under this curve?
% (Insert figure of area under parabola x^2)

\section{Introduction}

What an odd question, after all. The question, on the outset, seems as un-mysterious as can be. What is the area of this "shape"? 

It's a well-defined question too. If I needed to paint the inside of this shape, and I need a certain amount of paint per unit area, it makes perfect sense to ask (and get an answer for) how much paint I will need.

% More stuff here? (Maybe not)

In this second chapter, we'll do a few things. 

\begin{itemize}
\item We'll first think about how we find the area of "standard" shape. (rectangles, triangles, circles, etc.)
\item Using that foundation, we'll come up with a clever approximation to the area under a curved surface. 
\item We'll then show that our "approximation" is not approximate at all - it actually will give us the correct answer!
\item Finally, we'll show that the core idea of this problem goes far beyond area, and impacts an enormous fields in science and engineering.
\end{itemize}


\section{Areas we know and love}

What is the area of a rectangle? You might have heard it before. If one pair of sides is each of length $a$, and the other pair of sides is each of length $b$, then the area is given by their product, $ab$. \footnote{Sometimes, when looking at the piece of paper, the rectangle is standing upright. We then call the horizontal size the "base" (often labeled $b$), and the "height" (often labeled $h$).}

% TODO: Insert standard rectangle area figure

% TODO: Mention something about units? Or at least a footnote that explains the issue, and asks us not to worry about them for now.

What about the area of a triangle? Suppose for now it is a right triangle\footnote{TODO: Explain right triangles, right angles, etc}, as in the figure below. If one of the "short" sides has length $a$, and the other has length $b$, then its area is given by $\frac{1}{2}ab$. 

% TODO: Insert standard right triangle figure area

How could we have figured this out from our previous information? Consider that, if I made a copy of the triangle, and flipped it around and placed it as in the figure below,  I would have the same rectangle that I had previously! Therefore, a triangle of this form is quite literally "half" of our old rectangle, and the area is halved accordingly.

% TODO: Possibly explain areas of parallelogram, non-right triangle, circle


\section{Approximating new areas}

So let's return to the question we had at the beginning.

We form a strange "shape", colored in the figure below. The bottom and right edges are straight lines, but part of our shape's edge is a \textit{curve} formed by the function $f(x) = x^2$. What is our "shape"'s area?

% TODO: Insert title figure


Let's say I didn't actually care about the exact answer - I just wanted an approximation. We can use the shapes that we know! Suppose we lined up a bunch of rectangles, as in the figure below. If we add up the area of all of those rectangles, we wouldn't have the exact area we're looking for - but we'd be pretty close!

% TODO: Insert Riemann sum figure


What if we used a different number of rectangles to estimate our area? From the figure below, you can guess what happens. As we jam more and more rectangles around our shape, it looks as though, if we were to add up the area of those rectangles, we'd get a closer and closer estimate of the area of our shape!

% TODO: Insert figure with progressively better Riemann approximations (2, 4, 8, 32)

Let's actually try it. First, let's start with using two rectangles, like in the first figure above. Each rectangle has a "base" of length $1/2$. The first rectangle has a height of $f(1/2) = (1/2)^2 = 1/4$, while the second rectangle has a height of $f(1) = 1^2 = 1$. 

Let's start making up some variable names for us to use. (These will be important later on.) Let $A_{i}$ be the area of the $i$-th rectangle. Let $b_{i}$ and $h_{i}$ be the length of the $i$-th rectangle's base and height, respectively. 

Then,  

\begin{align*}
\text{Area estimate} &= A_1 + A_2 \\
&= b_1 h_1 + b_2 h_2
&= (1/2)
\end{align*}






















%%%%%% Below is the old stuff

\section{Numbers in Our World}

\subsection{}

The position of an object, the amount of fluid in a bucket, the concentration of something in something else (say, of oxygen in the air), the force generated by a rocket's engines -- these are all things describable with some combination of numbers and "units" (which tell us "numbers of what"). 

% TODO: Clarify what a reference point is.

Some fitting examples of such quantities, respectively, could be "5.266 feet to the right of that telephone pole"\footnote{This quantity also has a "reference point". The quantity "5.266 feet" alone wouldn't very useful in describing where our object is.}, "15.2 grams per liter (or 15.2 g/L)", and 7,500,000 pounds.\footnote{Saturn V first stage thrust, \url{https://en.wikipedia.org/wiki/Saturn_V}}

% TODO: Possibly clarify that the telephone pole is a reference point. 
% TODO: That footnote looks an awful lot like a square.  


Moreover, unlike some other measurements -- such as the number of balls in an urn, or the number of times the word "the" appears in this book -- we can describe our previous quantities with a much greater set of numbers than the "counting numbers" $(1, 2, 3, ...)$. We can't really have 1.6 balls in an urn (even if a ball were cut up, it's not really a "ball" anymore), but it makes sense to have 1.6 liters of fluid in a bucket. We could also have 1.7 liters in that bucket, or pretty much any quantity in between 1.6 and 1.7 liters (1.6333, $\frac{1 + \sqrt{5}}{2}$\footnote{This quantity is called the "golden ratio", and it appears in a lot of interesting places. Look it up!}, etc.) - any such value would make sense here. \footnote{Some of you may be thinking about the fact that fluid is just a bunch of atoms and molecules, and that those aren't really indivisible. But our "infinitely divisible fluid" is a nice approximation for now.}

% TODO: Again, footnotes are numbers. This is annoying and confusing. 

\textbf{Exercise 1:} 



\section{Changing the Numbers}

%Say I started a stopwatch, and then immediately left from Los Angeles toward San Francisco (about 350 miles away), arriving at time $t = 5$ hours. I moved across

% Many such quantities have the ability to change over time, and this change is usually described through another quantity. 

% For example, for something to move from point A to point B, it must have, at some point, had some nonzero speed. Let's say

% TODO: Really use the term "nonzero"?
% TODO: Insert the fluid example





Say I were at the beginning of a long, straight road. I start my stopwatch, and proceed to drive along that road at a certain \emph{speed} (say, 70 miles per hour). 

I now have enough information to say where I am on the road at \emph{any} given time. For example, 

\begin{itemize}
\item At $t = 1$ hour, I am 70 miles down the road.
\item At $t = 3$ hours, I am 210 miles down the road.
\item At $t = 0.6525$ hours, I am 45.675 miles down the road.
\end{itemize}

The speed is essentially "rate of change information" for position. The name of the unit itself is suggestive: We are traveling at 70 miles \emph{per hour}.




% insert stuff here...






\end{document}
